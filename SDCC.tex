\documentclass[conference]{IEEEtran}
\IEEEoverridecommandlockouts

\usepackage{lettrine}

\usepackage[pdfpagelabels]{hyperref}
\usepackage{url}

\usepackage[sorting=none]{biblatex}
\addbibresource{SDCC.bib}

%\usepackage{listings}
%\lstset{
%  language=go,
%  basicstyle=\ttfamily
%}

\begin{document}

\title{SDCC}

\author{
\IEEEauthorblockN{Daniele Ferrarelli\IEEEauthorrefmark{1},
Marco Ferri\IEEEauthorrefmark{2},
Lorenzo Valeriani\IEEEauthorrefmark{3}}
\IEEEauthorblockA{\textit{Dipartimento di Ingegneria Civile e Ingegneria
Informatica}\\
\textit{Università degli studi di Roma ``Tor Vergata''}\\
Roma, Italia\\
\IEEEauthorrefmark{1}daniele.ferrarelli@alumni.uniroma2.eu\\
\IEEEauthorrefmark{2}marco.ferri.98@alumni.uniroma2.eu\\
\IEEEauthorrefmark{3}lorenzo.valeriani.459326@alumni.uniroma2.eu}
}

\pdfinfo{
  /Title    (LPN: Log Partition Network)
  /Author   (Daniele Ferrarelli, Marco Ferri, Lorenzo Valeriani)
  /Creator  (Daniele Ferrarelli, Marco Ferri, Lorenzo Valeriani)
  /Producer (Daniele Ferrarelli, Marco Ferri, Lorenzo Valeriani)
  /Subject  (LPN)
  /Keywords (LPN)
}

\maketitle

\begin{abstract}
In questo documento si illustra una soluzione implementativa di uno storage chiave-valore distribuito per edge computing.
\end{abstract}

\begin{IEEEkeywords}
sistemi distribuiti, peer-to-peer, edge computing, cloud computing
\end{IEEEkeywords}

\section{Introduzione}
\lettrine{\textbf{L}}{\textbf{e}} reti peer-to-peer rappresentano una soluzione comune per il problema dello storage distribuito. Tra queste, quelle strutturate si prestano particolarmente alle tipologie di storage chiave-valore. Attraverso tecniche come il consistent hashing, si supportano le operazioni di ricerca delle informazioni in modo efficiente %aggiungere citazione coatta
L'edge computing è un paradigma di computazione che estende il cloud tradizionale con le funzionalità di computazione e storage offerte da nodi localizzati ai bordi della rete e quindi vicino agli utenti. La soluzione proposta sfrutta una rete peer-to-peer strutturata per indicizzare le risorse distribuite sui vari nodi edge, traendo i benefici di entrambe le realtà. Kademlia Raft (bisogna dire che è fault tollerant) e tanto cloud
\section{Topologia}
Qui bisogna dire che si usa indexing
\subsection{Rete di indexing}
\subsection{Rete di replicazione}
\section{Storage locale}
\section{Consistenza e semantica di errore}
\section{Migrazione}
\section{Integrazione con il cloud}
\section{Limitazioni}
\section{Testing}
\section{Manuale d'uso}
\subsection{Installazione}
\subsection{Configurazione}
\subsection{Esecuzione}

Sottosezione del documento in cui inseriremo la relazione del progetto di sistemi
distribuiti e cloud computing.

\textbf{Ricorda che in caso di offloading su cloud rimane il campo value della DHT su cui mettere se è su Dynamo o S3}

\textbf{Ricorda di dire che le porte della DHT possono anche non essere statiche per ogni nodo, ma le abbiamo messe così per esporle sui docker}

\printbibliography

\end{document}
