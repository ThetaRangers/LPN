\documentclass[conference]{IEEEtran}
\IEEEoverridecommandlockouts

\usepackage{lettrine}

\usepackage[pdfpagelabels]{hyperref}
\usepackage{url}

\usepackage[sorting=none]{biblatex}
\addbibresource{SDCC.bib}

\usepackage{listings}
\lstset{
  language=go,
  basicstyle=\ttfamily
}

\begin{document}

\title{SDCC}

\author{
\IEEEauthorblockN{Daniele Ferrarelli\IEEEauthorrefmark{1},
Marco Ferri\IEEEauthorrefmark{2},
Lorenzo Valeriani\IEEEauthorrefmark{3}}
\IEEEauthorblockA{\textit{Dipartimento di Ingegneria Civile e Ingegneria
Informatica}\\
\textit{Università degli studi di Roma ``Tor Vergata''}\\
Roma, Italia\\
\IEEEauthorrefmark{1}daniele.ferrarelli@alumni.uniroma2.eu\\
\IEEEauthorrefmark{2}marco.ferri.98@alumni.uniroma2.eu\\
\IEEEauthorrefmark{3}lorenzo.valeriani.459326@alumni.uniroma2.eu}
}

\pdfinfo{
  /Title    (SDCC)
  /Author   (Daniele Ferrarelli, Marco Ferri, Lorenzo Valeriani)
  /Creator  (Daniele Ferrarelli, Marco Ferri, Lorenzo Valeriani)
  /Producer (Daniele Ferrarelli, Marco Ferri, Lorenzo Valeriani)
  /Subject  (SDCC)
  /Keywords (SDCC)
}

\maketitle

\begin{abstract}
Questo documento è un modello. \LaTeX
\end{abstract}

\begin{IEEEkeywords}
sistemi distribuiti, peer-to-peer, edge computing, cloud computing
\end{IEEEkeywords}

\section{Sezione}

\lettrine{\textbf{I}}{\textbf{nizio}} documento in cui inseriremo la relazione del
progetto di sistemi distribuiti e cloud computing.\cite{foo}

\subsection{Sottosezione}

Sottosezione del documento in cui inseriremo la relazione del progetto di sistemi
distribuiti e cloud computing.

\textbf{Ricorda che in caso di offloading su cloud rimane il campo value della DHT su cui mettere se è su Dynamo o S3}

\printbibliography

\end{document}
